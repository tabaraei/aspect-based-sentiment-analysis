
\section{Concluding remarks}\label{sec4}

The following remarks can be concluded from this study:
\begin{itemize}
    \item Using the \texttt{SemEval2014Dataset} for training the \texttt{BornClassifier} yielded better results compared to the \texttt{AmazonReviewsDataset} in sentiment analysis, suggesting the importance of domain-specific training data for better generalization.

    \item Aspect detection, although effective for simple aspects, struggled with complex noun phrases due to reliance on unigram tokenization. In addition, the \texttt{spaCy} model itself was not sufficient to find all the candidate aspects according to POS tags.
    
    \item The lack of advanced techniques such as lemmatization or n-grams due to OOV concerns may have hindered performance.

    \item In both sentiment analysis and aspect-based sentiment analysis (ABSA), the \texttt{BornClassifier} has demonstrated superior performance compared to the NLTK and RoBERTa models.
\end{itemize}

For future work, a key priority would be to develop specialized datasets that do not rely on sentiment mapping, allowing for a more accurate ground-truth set. Moreover, addressing the issue of OOV tokens is crucial, as this would enable the use of advanced normalization techniques and the application of n-grams for improved performance.\newpage